
\documentclass[a4paper]{scrreprt}
 
\usepackage[german]{babel}
\usepackage[utf8]{inputenc}
\usepackage[T1]{fontenc}
\usepackage{ae}

\usepackage[scaled]{uarial}
\renewcommand\familydefault{\sfdefault} 

\usepackage[onehalfspacing]{setspace}
\usepackage[scaled]{helvet}
\renewcommand*\familydefault{\sfdefault}

\usepackage[T1]{fontenc}
\usepackage{glossaries}
\usepackage{graphicx}
\usepackage[bookmarks,bookmarksnumbered]{hyperref}


\setcounter{tocdepth}{1} 
\setcounter{secnumdepth}{2} 

\makenoidxglossaries
\newglossaryentry{App}
{ 	name=App,
	plural=Apps,
	description={Als Mobile App (auf Deutsch meist in der Kurzform die App, eine Abkürzung für den Fachbegriff Applikation) wird eine Anwendungssoftware für Mobilgeräte beziehungsweise mobile Betriebssysteme bezeichnet}
}

\newglossaryentry{Desktop Anwendung}
{	name=Desktop Anwendung,
	plural=Desktop Anwendungen,
	description={Als Desktop Anwendungen (auch Anwendungsprogramm, kurz Anwendung oder Applikation; englisch application software, kurz App) werden Computerprogramme bezeichnet, die genutzt werden, um eine nützliche oder gewünschte nicht systemtechnische Funktionalität zu bearbeiten oder zu unterstützen. Sie dienen der „Lösung von Benutzerproblemen“}
}

\newglossaryentry{Drag and Drop}
{	name=Drag and Drop,
	description={Drag and Drop, oft auch Drag’n’Drop, deutsch „Ziehen und Ablegen“, ist eine Methode zur Bedienung grafischer Benutzeroberflächen von Rechnern durch das Bewegen grafischer Elemente mittels eines Zeigegerätes. Ein Element wie z. B. ein Piktogramm kann damit gezogen und über einem möglichen Ziel losgelassen werden. Dieses kann zum Beispiel markierter Text oder das Symbol einer Datei sein. }
}

\newglossaryentry{NFC}
{ 	name=NFC,
	description={Die Nahfeldkommunikation (Near Field Communication, abgekürzt NFC) ist ein auf der RFID-Technik basierender internationaler Übertragungsstandard zum kontaktlosen Austausch von Daten per elektromagnetischer Induktion mittels loser gekoppelter Spulen über kurze Strecken von wenigen Zentimetern und einer Datenübertragungsrate von maximal 424 kBit/s.}
}

\newglossaryentry{Bluetooth}
{	name=Bluetooth,
	description={Bluetooth ist ein in den 1990er Jahren durch die Bluetooth Special Interest Group (SIG) entwickelter Industriestandard gemäß IEEE 802.15.1 für die Datenübertragung zwischen Geräten über kurze Distanz per Funktechnik (WPAN). Dabei sind verbindungslose sowie verbindungsbehaftete Übertragungen von Punkt zu Punkt und Ad-hoc- oder Piconetze möglich}
}
 
\begin{document}
 

\begin{titlepage}
\begin{figure}[h]
	\vspace{-4cm}
	\hspace{-2cm}
	\includegraphics[ width=0.3\textwidth]{Kit_Logo}
	\label{fig:Aufg03_1}
\end{figure}
	\vspace{1.5cm}
	\centering
	\includegraphics[width=0.5\textwidth]{myMD_Logo}\par\vspace{0.5cm}
	{\itshape\huge myMD \par}
	\vspace{2cm}
	{\scshape\Large Pflichtenheft\par}
	\vspace{2cm}
	{\Large\itshape Philipp Pelcz, Philipp Karcher, Jan-Luca Vettel\par}
	\vfill
	supervised by\par
	Marc Aurel Kiefer

	\vfill

% Bottom of the page
	{\large \today\par}
\end{titlepage}
 

% Platzierung des Inhaltsverzeichnisses
\tableofcontents
\addtocontents{toc}{\protect\enlargethispage{10cm}}



\chapter{Zielbestimmung}

\section{Einleitung}
TODO: eine coole 1 seitige Einleitung schreiben
 
\section{Musskriterien (MK)}
\subsection{Patientenseitige Datenübertragung}
\begin{tabular}{lll}
[MK1010] & \multicolumn{2}{p{12cm}}  {Die Arztbriefe können von der \gls{Desktop Anwendung} auf die myMD \gls{App} des Patienten übertragen werden.} \\
{[MK1020]} & \multicolumn{2}{p{12cm}}  {Die Übertragung findet verschlüsselt über \gls{Bluetooth} statt.} \\
\end{tabular}

\subsection{Darstellung}
\begin{tabular}{lll}
[MK2010] & \multicolumn{2}{p{12cm}}  {Arztbriefe werden chronologisch absteigend im Tab Übersicht dargestellt.} \\
{[MK2020]} & \multicolumn{2}{p{12cm}}  {Der Nutzer kann nicht benötigte/nicht gewollte Arztbriefe löschen.} \\
{[MK2030]} & \multicolumn{2}{p{12cm}}  {Aus einem dargestellten Arztbrief kann man immer die Diagnose, verordnete Medikament, das Datum und den Namen des Arztes entnehmen.} \\
\end{tabular}

\subsection{Einstellungen}
\begin{tabular}{lll}
[MK3010] & \multicolumn{2}{p{12cm}}  {Der Nutzer kann ein eigenes Profil anlegen, welches Daten wie den Namen, Blutgruppe, Versicherungsnummer und Allergien enthält.} \\
{[MK3020]} & \multicolumn{2}{p{12cm}}  {Die myMD \gls{App} wird in Deutsch angeboten.} \\

\end{tabular}

\subsection{Desktop Anwendung}
\begin{tabular}{lll}
{[MK4010]} & \multicolumn{2}{p{12cm}}  {Die Desktop Anwendung kann die Geräte in der Nähe anzeigen.} \\
{[MK4020]} & \multicolumn{2}{p{12cm}}  {Der Arzt kann unter den Geräten in der Nähe das Handy des Patienten als Empfänger auswählen.} \\
{[MK4030]} & \multicolumn{2}{p{12cm}}  {Die Dateien können entweder per \gls{Drag and Drop} oder über einen Explorer in die Desktop Anwendung geladen werden.} \\
{[MK4040]} & \multicolumn{2}{p{12cm}}  {Die Daten werden auf Knopfdruck an das Handy des Patienten gesendet.} \\
\end{tabular}

\subsection{Kompabilität}
\begin{tabular}{lll}
[MK5010] & \multicolumn{2}{p{12cm}}  {Die \gls{Desktop Anwendung}wird von Microsoft Windows 10 unterstützt.} \\
{[MK5020]} & \multicolumn{2}{p{12cm}}  {Die myMD \gls{App} wird von Android 6.0 (und höher) unterstützt.} \\

\end{tabular}
 
\section{Wunschkriterien (WK)}
\subsection{Patientenseitige Datenübertragung}
\begin{tabular}{lll}
[MK1010] & \multicolumn{2}{p{12cm}}  {Ein Arztbriefe kann von der myMD \gls{App} des Patienten auf die \gls{Desktop Anwendung} übertragen werden.} \\
{[MK1020]} & \multicolumn{2}{p{12cm}}  {Der Patient kann mehrere Arztbriefe aufeinmal senden.} \\
{[MK1030]} & \multicolumn{2}{p{12cm}}  {\gls{NFC} steht als weiter Übertragunsmöglichkeit zur Verfügung.} \\
{[MK1040]} & \multicolumn{2}{p{12cm}}  {Der Patient wird vor dem Senden von sensiblen Daten darauf hingewiesen, dass er sensible Daten versendet.} \\
\end{tabular}

\subsection{Darstellung}
\begin{tabular}{lll}
[MK2010] & \multicolumn{2}{p{12cm}}  {Eingenommene Medikamente werden in einem extra Tab, chronologisch ansteigend sortiert, dargestellt.} \\
{[MK2020]} & \multicolumn{2}{p{12cm}}  {Laborwerte des Patienten werden in einem extra Tab, chronologisch absteigend sortiert, dargestellt.} \\
{[MK2030]} & \multicolumn{2}{p{12cm}}  {Ein Arztbrief kann Röntgenbilder enthalten und die myMD \gls{App} kann diese originalgetreu und innerhalb des Arztbriefes darstellen.} \\
{[MK2040]} & \multicolumn{2}{p{12cm}}  {Man kann die Arztbriefe nach eigenen Kriterien (Arzt, Krankheit o.ä.) gruppieren.} \\
{[MK2020]} & \multicolumn{2}{p{12cm}}  {Es gibt eine Suchfunktion, die Daten in einem Arztbrief sucht.} \\
\end{tabular}

\subsection{Einstellungen}
\begin{tabular}{lll}
[WK3010] & \multicolumn{2}{p{12cm}}  { Auf einer myMD \gls{App} können mehrere Nutzer verwaltet werden.} \\
{[WK3020]} & \multicolumn{2}{p{12cm}}  {Die myMD \gls{App} wird zusätzlich auch in Englisch angeboten.} \\
{[WK3030]} & \multicolumn{2}{p{12cm}}  {Der Nutzer kann einzelne Arztbriefe oder ganze Gruppen als sensibel markieren.} \\
{[WK3040]} & \multicolumn{2}{p{12cm}}  {Die myMD \gls{App} ermöglicht, dass man sich zu regelmäßigen Arztterminen (z.B. Zahnarzt,Augenarzt) erinnern lässt.} \\

\end{tabular}

\subsection{Desktop Anwendung}
\begin{tabular}{lll}
{[WK4010]} & \multicolumn{2}{p{12cm}}  {Die Versicherungsnummer, die in einem Arztbrief auf dem Computer des Arztes eingetragen ist, wird vor dem Senden mit der Versicherungsnummer, die in dem Profil des myMD Nutzers, verglichen und nur bei Übereinstimmung wird der Arztbrief gesendet.} \\


\end{tabular}

\subsection{Kompabilität}
\begin{tabular}{lll}
{[WK5010]} & \multicolumn{2}{p{12cm}}  {Die myMD \gls{App} wird von IOS 10 (und höher) unterstützt.} \\
{[WK5020]} & \multicolumn{2}{p{12cm}}  {Die \gls{Desktop Anwendung}wird zusätzlich von macOS unterstützt.} \\

\end{tabular}
 
\section{Abgrenzungskriterien (AK)}
\begin{tabular}{ll}
[AK0010] &  Die Anwendung selbst stellt keine medizinischen Diagnosen. \\
{[AK0020]} &  Es werden keine Daten auf einem Server oder in einer Cloud gespeichert. \\
{[AK0030]} &  Die Anwendung stellt keinen Ersatz zu einem Arzttermin dar. \\
{[AK0040]} &  Die Anwendung stellt keinen Ersatz zu einer Versichertenkarte dar. \\
{[AK0050]} &  Es gibt keine Möglichkeit zur Terminvereinbarung. \\
{[AK0060]} &  Der Nutzer kann einen Arztbrief nicht editieren. \\
\end{tabular}
 
\chapter{Produkteinsatz}

\section{Einsatzgebiet}
myMD ist bei Arztebesuchen aller Art und in jeder anderen Situation in der man auf seine medizinschen Daten zugreifen möchte einsetzbar.

\section{Produktumgebung}
TODO: Eingeben

\section{Betriebsbedingungen}
\begin{tabular}{lll}
\gls{Bluetooth} & \multicolumn{2}{p{12cm}}{Sowohl das Handy des Patienten als auch der Computer des Arztes müssen mindestens über \gls{Bluetooth} 4.0 verfügen.} \\
{\gls{App} Betriebssystem} & \multicolumn{2}{p{12cm}}  {Auf dem Handy des Patienten muss entweder Android 6.0 (oder höher) oder IOS 10.0 (oder höher) installiert sein.}\\
{Sendetool Betriebssystem} & \multicolumn{2}{p{12cm}} {Auf dem Computer des Arztes muss Windows 10 installiert sein.} \\
\end{tabular} 
 
\section{Zielgruppe}
Die Anwendung richtet sich an zwei verschiedene Zielgruppen:  \\\\
\begin{tabular}{lll}
PATIENTEN &  \multicolumn{2}{p{12cm}}{Dies sind alle Menschen die momentan oder in der Vergangenheit eine medizinische Behandlung oder Beratung in Anspruch genommen haben. Ihr Ziel ist es alle Daten darüber auf ihrem Handy zu sammeln und einsehen zu können.}\\
ÄRZTE &  \multicolumn{2}{p{12cm}}{Diese bieten medizinische Behandlung und Beratung an. Die Anwendung erlaubt ihnen alle relevanten medizinischen Daten über einen Patienten von diesem zu erhalten und ihm selbst erstellte Daten zu übermitteln.}  \\
\end{tabular}

\chapter{Produktfunktionen}
TODO: Funktionen überlegen und eintragen, dazu am besten an Kriterien halten

\section{Grundfunktionen}
TODO: Eintragen

\section{Optionale Funktionen}
TODO: Eintragen

\section{Produktleistungen}
TODO: Eintragen

\section{Qualitäts-Zielbestimmungen}
TODO: Eintragen

\chapter{Produktdaten}
TODO: Eintragen

\chapter{Systemmodelle}
\section{Systemarchitektur}
TODO: Eintragen
\section{Szenarien}
\subsection{Szenario: Patient lädt sich neue Daten auf myMD}
Max Mustermann ist ein neuer Nutzer von myMD. Deshalb hat er auf seinem Konto auch noch keine Arztbriefe eingetragen. Also geht er zu seinem Hausarzt, um seine Krankenakte für die myMD \gls{App} zu erhalten. Der Arzt öffnet dafür seine Praxissoftware und exportiert alle Arztbriefe, die zu Max gehören, und lädt diese in das \gls{Desktop Anwendung} von myMD. Nachdem die Arztbriefe in dem Sendetool angezeigt werden, klickt der Arzt auf den Button "Daten senden". Nun wechselt Max auf seiner myMD \gls{App} in das Senden-Menü und klickt "Daten empfangen". Auf Max's myMD \gls{App} erscheint jetzt die Nachricht, dass die Daten an ihn gesendet werden. Sobald der Sendeprozess beendet ist, kann der Arzt das \gls{Desktop Anwendung} von myMD schließen und sich dem nächsten Patienten widmen. Max hat jetzt all seine Arztbriefe auf seinem Handy und kann sich diese jederzeit in der myMD \gls{App} anzeigen lassen. Sie werden in dem Tab "Übersicht", chronologisch sortiert, dargestellt. Wenn er genauere Informationen zu einem bestimmten Arztbrief haben möchte, kann er auf den Arztbrief klicken und dieser wird nun auf dem ganzen Bildschirm, mit allen Informationen, dargestellt. So hat Max jederzeit eine Übersicht

\subsection{Eigene Daten ansehen und editieren}
Greg leidet seit einigen Tagen unter starken Magenbeschwerden. Sein Hausarzt findet keine Ursache für die Beschwerden und überweist ihn deshalb an den Spezialisten Dr. Haus und verschreibt ihm etwas gegen die Schmerzen. Dr. Haus glaubt die Ursache gefunden zu haben, diese lässt sich aber nur durch eine Operation beseitigen, für die er Greg an einen Chirurgen überweist. Bevor dem Eingriff zustimmt, will er sich noch eine zweite Meinung einholen. Dafür lässt er sich von seinem Hausarzt einen weiteren Spezialisten empfehlen, der jedoch keine medizinische Ursache für das Problem erkennen kann. Er empfiehlt Greg eine psychotherapeutische Behandlung und verschreibt ihm zur Überbrückung ein stärkeres Schmerzmittel. Bei so vielen verschieden Ärzten und Diagnosen hat Greg nun etwas den Überblick verloren. Glücklicherweise hat er bei seinen Arztbesuchen myMD verwendet und hat nun alle relevanten Daten auf seinem Handy gesammelt und kann sich in der App einen Überblick verschaffen. Letztendlich entscheidet sich Greg es mit der Psychotherapie zu versuchen und entfernt die nun überflüssigen schwächeren Schmerzmittel aus der Liste seiner Medikationen in myMD. seiner Krankengeschichte.

\section{Anwendungsfälle}
TODO: Diagramme erstellen und Eintragen
\section{Benutzungsoberfläche}
TODO: GUI Entwürfe  einbinden
\begin{figure}[ht]
  \centering
  \rule{8cm}{6cm}
  \caption{Dies könnte ein Bild der Benutzungsoberfläche sein}
\end{figure}

\chapter{Testfälle}
\section{Basis Testfälle}
TODO: Eintragen
\section{Erweiterte Testfälle}
TODO: Eintragen
\section{TestSzenarien}
TODO: Eintragen
 

\chapter{Entwicklungsumgebung}
 
\section{Software}
\begin{tabular}{lll}
Entwicklungsumgebung &  \multicolumn{2}{p{12cm}}{Visual Studio 2017, Visual Studio for Mac}\\
Dokumentation &  \multicolumn{2}{p{12cm}}{Microsoft PowerPoint, LaTeX}  \\
GUI Entwürfe & \multicolumn{2}{p{12cm}}{Sketch} \\
Versionierung & \multicolumn{2}{p{12cm}}{Git, GitLab} \\
\end{tabular}

 
\section{Hardware}
\begin{tabular}{lll}
Omen by HP &  \multicolumn{2}{p{12cm}}{Intel Core i7-6700HQ CPU @ 2.60GHz \newline 16GB DDR3 RAM  \newline 64-Bit Microsoft Windows 10}\\
Custom Desktop PC &  \multicolumn{2}{p{12cm}}{Intel Xeon CPU E3-1231 v3 @ 3.40GHz \newline 8GB DDR3 RAM  \newline 64-Bit Microsoft Windows 10}\\
Apple MacBook Pro &  \multicolumn{2}{p{12cm}}{Intel Core i5-i5-2435M CPU @ 2.40GHz \newline 16GB DDR3 RAM  \newline Apple macOS 10.13}\\
\end{tabular}
 
 
\printnoidxglossaries

% Abbildungsverzeichnis
\listoffigures
 
\end{document}
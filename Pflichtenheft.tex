
\documentclass[a4paper]{scrreprt}
 
\usepackage[german]{babel}
\usepackage[utf8]{inputenc}
\usepackage[T1]{fontenc}
\usepackage{ae}
\usepackage[bookmarks,bookmarksnumbered]{hyperref}
 
\begin{document}
 
\title{myMD - Pflichtenheft}

\author{Philipp Pelcz \and Philipp Karcher \and Jan-Luca Vettel}
\maketitle
 

% Platzierung des Inhaltsverzeichnisses
\tableofcontents
 
\chapter{Zielbestimmung}
Dieses Kapitel dient der Bestimmung von Zielen und nicht für deren Verwendung
notwendige Funktionen.

\section{Einleitung}
Dies ist ein Produkt, um medizinische Daten aufzulisten
 
\section{Musskriterien (MK)}
Musskriterien: Für das Produkt unabdingbare Leistungen, die in jedem Fall
erfüllt werden müssen \footnote{gezwungen sein, etwas zu tun (Dies ist eine
beispielhafte Fußnote).}. Das System ist ohne diese Funktionen für seinen
gedachten Zweck nicht einsetzbar.
 
\section{Wunschkriterien (WK)}
Kannkriterien: Die Erfüllung der Kannkriterien ist erwünscht, jedoch nicht
unbedingt notwendig. Sie sollten nur angestrebt werden, falls noch ausreichend
Kapazitäten vorhanden sind.
 
\section{Abgrenzungskriterien (AK)}
Abgrenzungskriterien: Diese Kriterien sollen bewusst nicht erreicht werden.
 
\chapter{Produkteinsatz}
Der geplante Einsatz des Systems ist die Grundlage für Benutzungsoberfläche und
Qualitätsanforderungen.

\section{Einsatzgebiet}
Hier wird das Einsatzgebiet beschrieben
 
\section{Produktumgebung}
Ein Pflichtenheft wird bspw. in einer IT-Abteilung genutzt.

\section{Betriebsbedingungen}
Betriebsbedingungen: Die Betriebsbedingungen spezifiziert die physikalische
Umgebung des Systems, die tägliche Betriebszeit, und ob das System ständiger
Beobachtung durch Bediener ausgesetzt ist, oder ein unbeaufsichtigter Betrieb
beabsichtigt ist.
 
 
\section{Zielgruppe}
Die Zielgruppe besteht also aus Informatikern, die mit der Projektplanung
beauftragt wurden.

\chapter{Produktfunktionen}
Funktionalität: Spezifikation der einzelnen Produktfunktionen mit genauer und
detaillierter Beschreibung.

\begin{itemize}
  \item Typische Arbeitsabläufe
  \item Keine Angabe von typischen Verwaltungsfunktionen (CRUD \footnote{Create,
Read, Update, Delete}
\end{itemize}

\section{Grundfunktionen}
Hier stehen die Grundfunktionen

\section{Optionale Funktionen}
Hier stehen optionale Funktionen

\section{Produktleistungen}
Hier stehen die Produktleistungen

\section{Qualitäts-Zielbestimmungen}
TEXT...

\chapter{Produktdaten}

\chapter{Systemmodelle}
\section{Systemarchitektur}
\section{Szenarien}
\section{Anwendungsfälle}
\section{Benutzungsoberfläche}
Benutzungsoberfläche: grundlegende Anforderungen, Zugriffsrechte
 
\begin{figure}[ht]
  \centering
  \rule{8cm}{6cm}
  \caption{Dies könnte ein Bild der Benutzungsoberfläche sein}
\end{figure}

\chapter{Testfälle}
\section{Basis Testfälle}
\section{Erweiterte Testfälle}
\section{TestSzenarien}
 

\chapter{Entwicklungsumgebung}
 
\section{Software}
Software: Gibt an, welche Software zum Betrieb vorhanden sein muss. Eine
Aufteilung für Server und Client ist ggf. sinnvoll. Weiterhin sind unbedingt die
kleinsten benötigten Versionsnummern anzugeben.
 
\section{Hardware}
Hardware: Hardware-Anforderungen des Systems.
 
\section{Orgware}
Orgware: Angabe der organisatorische Rahmenbedingungen, die vor Projektstart
erfüllt sein müssen.
 
 
\chapter{Glossar}
Hier ist Platz für nicht im Pflichtenheft abgedeckte Themengebiete oder ein
Glossar.
 
% Abbildungsverzeichnis
\listoffigures
 
\end{document}

\documentclass[a4paper]{scrreprt}
 
\usepackage[german]{babel}
\usepackage[utf8]{inputenc}
\usepackage[T1]{fontenc}
\usepackage{ae}
\usepackage[bookmarks,bookmarksnumbered]{hyperref}

 
\begin{document}
 
\title{myMD - Pflichtenheft}

\author{Philipp Pelcz \and Philipp Karcher \and Jan-Luca Vettel}
\maketitle
 

% Platzierung des Inhaltsverzeichnisses
\tableofcontents
 
\chapter{Zielbestimmung}
Dieses Kapitel dient der Bestimmung von Zielen und nicht für deren Verwendung
notwendige Funktionen.

\section{Einleitung}
Dies ist ein Produkt, um medizinische Daten aufzulisten
 
\section{Musskriterien (MK)}
Musskriterien: Für das Produkt unabdingbare Leistungen, die in jedem Fall
erfüllt werden müssen \footnote{gezwungen sein, etwas zu tun (Dies ist eine
beispielhafte Fußnote).}. Das System ist ohne diese Funktionen für seinen
gedachten Zweck nicht einsetzbar.
 
\section{Wunschkriterien (WK)}
Kannkriterien: Die Erfüllung der Kannkriterien ist erwünscht, jedoch nicht
unbedingt notwendig. Sie sollten nur angestrebt werden, falls noch ausreichend
Kapazitäten vorhanden sind.
 
\section{Abgrenzungskriterien (AK)}
%Abgrenzungskriterien: Diese Kriterien sollen bewusst nicht erreicht werden.
\begin{tabular}{ll}
[KA1010] &  Die Anwendung selbst stellt keine medizinischen Diagnosen. \\
{[KA1020]} &  Es werden keine Daten auf einem Server oder in einer Cloud gespeichert. \\
{[KA1030]} &  Die Anwendung stellt keinen Ersatz zu einem Arzttermin dar. \\
{[KA1040]} &  Die Anwendung stellt keinen Ersatz zu eine rVersichertenkarte dar. \\
{[KA1050]} &  Es gibt keine Möglichkeit zur Terminvereinbarung. \\
\end{tabular}

\chapter{Produkteinsatz}
%Der geplante Einsatz des Systems ist die Grundlage für Benutzungsoberfläche und Qualitätsanforderungen.

\section{Einsatzgebiet}
myMD ist bei Arztebesuchen aller Art und in jeder anderen Situation in der man auf seine medizinschen Daten zugreifen möchte einsetzbar.
 
\section{Produktumgebung}
Ein Pflichtenheft wird bspw. in einer IT-Abteilung genutzt.

\section{Betriebsbedingungen}
%Betriebsbedingungen: Die Betriebsbedingungen spezifiziert die physikalische Umgebung des Systems, die tägliche Betriebszeit, und ob das System ständiger Beobachtung durch Bediener ausgesetzt ist, oder ein unbeaufsichtigter Betrieb beabsichtigt ist.
 
 
\section{Zielgruppe}
%Die Zielgruppe besteht also aus Informatikern, die mit der Projektplanung beauftragt wurden.
Die Anwendung richtet sich an zwei verschiedene Zielgruppen:  \\\\
\begin{tabular}{lll}
PATIENTEN &  \multicolumn{2}{p{12cm}}{Dies sind alle Menschen die momentan oder in der Vergangenheit eine medizinische Behandlung oder Beratung in Anspruch genommen haben. Ihr Ziel ist es alle Daten darüber auf ihrem Handy zu sammeln und einsehen zu können.}\\
ÄRZTE &  \multicolumn{2}{p{12cm}}{Diese bieten medizinische Behandlung und Beratung an. Die Anwendung erlaubt ihnen alle relevanten medizinischen Daten über einen Patienten von diesem zu erhalten und ihm selbst erstellte Daten zu übermitteln.}  \\
\end{tabular}

\chapter{Produktfunktionen}
Funktionalität: Spezifikation der einzelnen Produktfunktionen mit genauer und detaillierter Beschreibung.

\begin{itemize}
  \item Typische Arbeitsabläufe
  \item Keine Angabe von typischen Verwaltungsfunktionen (CRUD \footnote{Create,
Read, Update, Delete}
\end{itemize}

\section{Grundfunktionen}
Hier stehen die Grundfunktionen

\section{Optionale Funktionen}
Hier stehen optionale Funktionen

\section{Produktleistungen}
Hier stehen die Produktleistungen

\section{Qualitäts-Zielbestimmungen}
TEXT...

\chapter{Produktdaten}

\chapter{Systemmodelle}
\section{Systemarchitektur}
\section{Szenarien}
Szenario: Greg leidet seit einigen Tagen unter starken Magenbeschwerden. Sein Hausarzt findet keine Ursache für die Beschwerden und überweist ihn deshalb an den Spezialisten Dr. Haus und verschreibt ihm etwas gegen die Schmerzen. Dr. Haus glaubt die Ursache gefunden zu haben, diese lässt sich aber nur durch eine Operation beseitigen, für die er Greg an einen Chirurgen überweist. Bevor dem Eingriff zustimmt, will er sich noch eine zweite Meinung einholen. Dafür lässt er sich von seinem Hausarzt einen weiteren Spezialisten empfehlen, der jedoch keine medizinische Ursache für das Problem erkennen kann. Er empfiehlt Greg eine psychotherapeutische Behandlung und verschreibt ihm zur Überbrückung ein stärkeres Schmerzmittel. Bei so vielen verschieden Ärzten und Diagnosen hat Greg nun etwas den Überblick verloren. Glücklicherweise hat er bei seinen Arztbesuchen myMD verwendet und hat nun alle relevanten Daten auf seinem Handy gesammelt und kann sich in der App einen Überblick verschaffen. Letztendlich entscheidet sich Greg es mit der Psychotherapie zu versuchen und entfernt die nun überflüssigen schwächeren Schmerzmittel aus der Liste seiner Medikationen in myMD.

\section{Anwendungsfälle}
\section{Benutzungsoberfläche}
Benutzungsoberfläche: grundlegende Anforderungen, Zugriffsrechte
 
\begin{figure}[ht]
  \centering
  \rule{8cm}{6cm}
  \caption{Dies könnte ein Bild der Benutzungsoberfläche sein}
\end{figure}

\chapter{Testfälle}
\section{Basis Testfälle}
\section{Erweiterte Testfälle}
\section{TestSzenarien}
 

\chapter{Entwicklungsumgebung}
 
\section{Software}
Software: Gibt an, welche Software zum Betrieb vorhanden sein muss. Eine
Aufteilung für Server und Client ist ggf. sinnvoll. Weiterhin sind unbedingt die
kleinsten benötigten Versionsnummern anzugeben.
 
\section{Hardware}
Hardware: Hardware-Anforderungen des Systems.
 
\section{Orgware}
Orgware: Angabe der organisatorische Rahmenbedingungen, die vor Projektstart
erfüllt sein müssen.
 
 
\chapter{Glossar}
Hier ist Platz für nicht im Pflichtenheft abgedeckte Themengebiete oder ein
Glossar.
 
% Abbildungsverzeichnis
\listoffigures
 
\end{document}
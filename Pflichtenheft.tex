
\documentclass[a4paper]{scrreprt}
 
\usepackage[german]{babel}
\usepackage[utf8]{inputenc}
\usepackage[T1]{fontenc}
\usepackage{ae}
\usepackage{glossaries}
\usepackage[bookmarks,bookmarksnumbered]{hyperref}

\makenoidxglossaries
\newglossaryentry{App}
{ 	name=App,
	plural=Apps,
	description={Als Mobile App (auf Deutsch meist in der Kurzform die App, eine Abkürzung für den Fachbegriff Applikation) wird eine Anwendungssoftware für Mobilgeräte beziehungsweise mobile Betriebssysteme bezeichnet}
}
 
\begin{document}
 
\begin{titlepage}

\title{myMD - Pflichtenheft}
\author{Philipp Pelcz \and Philipp Karcher \and Jan-Luca Vettel}

\end{titlepage} 

\maketitle
 

% Platzierung des Inhaltsverzeichnisses
\tableofcontents
 
\chapter{Zielbestimmung}
Dieses Kapitel dient der Bestimmung von Zielen und nicht für deren Verwendung
notwendige Funktionen.

\section{Einleitung}
Dies ist ein Produkt, um medizinische Daten aufzulisten
 
\section{Musskriterien (MK)}
Musskriterien: Für das Produkt unabdingbare Leistungen, die in jedem Fall
erfüllt werden müssen \footnote{gezwungen sein, etwas zu tun (Dies ist eine
beispielhafte Fußnote).}. Das System ist ohne diese Funktionen für seinen
gedachten Zweck nicht einsetzbar.
 
\section{Wunschkriterien (WK)}
Kannkriterien: Die Erfüllung der Kannkriterien ist erwünscht, jedoch nicht
unbedingt notwendig. Sie sollten nur angestrebt werden, falls noch ausreichend
Kapazitäten vorhanden sind.
 
\section{Abgrenzungskriterien (AK)}
\begin{tabular}{ll}
[KA1010] &  Die Anwendung selbst stellt keine medizinischen Diagnosen. \\
{[KA1020]} &  Es werden keine Daten auf einem Server oder in einer Cloud gespeichert. \\
{[KA1030]} &  Die Anwendung stellt keinen Ersatz zu einem Arzttermin dar. \\
{[KA1040]} &  Die Anwendung stellt keinen Ersatz zu einer Versichertenkarte dar. \\
{[KA1050]} &  Es gibt keine Möglichkeit zur Terminvereinbarung. \\
\end{tabular}
 
\chapter{Produkteinsatz}

\section{Einsatzgebiet}
myMD ist bei Arztebesuchen aller Art und in jeder anderen Situation in der man auf seine medizinschen Daten zugreifen möchte einsetzbar.

\section{Produktumgebung}


\section{Betriebsbedingungen}
\begin{tabular}{lll}
Bluetooth & \multicolumn{2}{p{12cm}}{Sowohl das Handy des Patienten als auch der Computer des Arztes müssen mindestens über Bluetooth 4.0 verfügen.} \\
{\gls{App} Betriebssystem} & \multicolumn{2}{p{12cm}}  {Auf dem Handy des Patienten muss entweder Android 6.0 (oder höher) oder IOS 10.0 (oder höher) installiert sein.}\\
{Sendetool Betriebssystem} & \multicolumn{2}{p{12cm}} {Auf dem Computer des Arztes muss Windows 10 installiert sein.} \\
\end{tabular} 
 
\section{Zielgruppe}
Die Anwendung richtet sich an zwei verschiedene Zielgruppen:  \\\\
\begin{tabular}{lll}
PATIENTEN &  \multicolumn{2}{p{12cm}}{Dies sind alle Menschen die momentan oder in der Vergangenheit eine medizinische Behandlung oder Beratung in Anspruch genommen haben. Ihr Ziel ist es alle Daten darüber auf ihrem Handy zu sammeln und einsehen zu können.}\\
ÄRZTE &  \multicolumn{2}{p{12cm}}{Diese bieten medizinische Behandlung und Beratung an. Die Anwendung erlaubt ihnen alle relevanten medizinischen Daten über einen Patienten von diesem zu erhalten und ihm selbst erstellte Daten zu übermitteln.}  \\
\end{tabular}

\chapter{Produktfunktionen}
Funktionalität: Spezifikation der einzelnen Produktfunktionen mit genauer und detaillierter Beschreibung.

\begin{itemize}
  \item Typische Arbeitsabläufe
  \item Keine Angabe von typischen Verwaltungsfunktionen (CRUD \footnote{Create,
Read, Update, Delete}
\end{itemize}

\section{Grundfunktionen}
Hier stehen die Grundfunktionen

\section{Optionale Funktionen}
Hier stehen optionale Funktionen

\section{Produktleistungen}
Hier stehen die Produktleistungen

\section{Qualitäts-Zielbestimmungen}
TEXT...

\chapter{Produktdaten}

\chapter{Systemmodelle}
\section{Systemarchitektur}
\section{Szenarien}
\subsection{Szenario: Patient lädt sich neue Daten auf myMD}
Max Mustermann ist ein neuer Nutzer von myMD. Deshalb hat er auf seinem Konto auch noch keine Arztbriefe eingetragen. Also geht er zu seinem Hausarzt, um seine Krankenakte für die myMD \gls{App} zu erhalten. Der Arzt öffnet dafür seine Praxissoftware und exportiert alle Arztbriefe, die zu Max gehören, und lädt diese in das Desktop-Sendetool von myMD. Nachdem die Arztbriefe in dem Sendetool angezeigt werden, klickt der Arzt auf den Button "Daten senden". Nun wechselt Max auf seiner myMD \gls{App} in das Senden-Menü und klickt "Daten empfangen". Auf Max's myMD \gls{App} erscheint jetzt die Nachricht, dass die Daten an ihn gesendet werden. Sobald der Sendeprozess beendet ist, kann der Arzt das Desktop-Sendetool von myMD schließen und sich dem nächsten Patienten widmen. Max hat jetzt all seine Arztbriefe auf seinem Handy und kann sich diese jederzeit in der myMD \gls{App} anzeigen lassen. Sie werden in dem Tab "Übersicht", chronologisch sortiert, dargestellt. Wenn er genauere Informationen zu einem bestimmten Arztbrief haben möchte, kann er auf den Arztbrief klicken und dieser wird nun auf dem ganzen Bildschirm, mit allen Informationen, dargestellt. So hat Max jederzeit eine Übersicht

\subsection{Eigene Daten ansehen und editieren}
Greg leidet seit einigen Tagen unter starken Magenbeschwerden. Sein Hausarzt findet keine Ursache für die Beschwerden und überweist ihn deshalb an den Spezialisten Dr. Haus und verschreibt ihm etwas gegen die Schmerzen. Dr. Haus glaubt die Ursache gefunden zu haben, diese lässt sich aber nur durch eine Operation beseitigen, für die er Greg an einen Chirurgen überweist. Bevor dem Eingriff zustimmt, will er sich noch eine zweite Meinung einholen. Dafür lässt er sich von seinem Hausarzt einen weiteren Spezialisten empfehlen, der jedoch keine medizinische Ursache für das Problem erkennen kann. Er empfiehlt Greg eine psychotherapeutische Behandlung und verschreibt ihm zur Überbrückung ein stärkeres Schmerzmittel. Bei so vielen verschieden Ärzten und Diagnosen hat Greg nun etwas den Überblick verloren. Glücklicherweise hat er bei seinen Arztbesuchen myMD verwendet und hat nun alle relevanten Daten auf seinem Handy gesammelt und kann sich in der App einen Überblick verschaffen. Letztendlich entscheidet sich Greg es mit der Psychotherapie zu versuchen und entfernt die nun überflüssigen schwächeren Schmerzmittel aus der Liste seiner Medikationen in myMD. seiner Krankengeschichte.

\section{Anwendungsfälle}
\section{Benutzungsoberfläche}
Benutzungsoberfläche: grundlegende Anforderungen, Zugriffsrechte
 
\begin{figure}[ht]
  \centering
  \rule{8cm}{6cm}
  \caption{Dies könnte ein Bild der Benutzungsoberfläche sein}
\end{figure}

\chapter{Testfälle}
\section{Basis Testfälle}
\section{Erweiterte Testfälle}
\section{TestSzenarien}
 

\chapter{Entwicklungsumgebung}
 
\section{Software}
\begin{tabular}{lll}
Entwicklungsumgebung &  \multicolumn{2}{p{12cm}}{Visual Studio 2017, Visual Studio for Mac}\\
Dokumentation &  \multicolumn{2}{p{12cm}}{Microsoft PowerPoint, LaTeX}  \\
GUI Entwürfe & \multicolumn{2}{p{12cm}}{Sketch} \\
Versionierung & \multicolumn{2}{p{12cm}}{Git, GitLab} \\
\end{tabular}

 
\section{Hardware}
\begin{tabular}{lll}
Omen by HP &  \multicolumn{2}{p{12cm}}{Intel Core i7-6700HQ CPU @ 2.60GHz \newline 16GB DDR3 RAM  \newline 64-Bit Microsoft Windows 10}\\
\end{tabular}
 
\section{Orgware}
Orgware: Angabe der organisatorische Rahmenbedingungen, die vor Projektstart
erfüllt sein müssen.
 
\printnoidxglossaries

% Abbildungsverzeichnis
\listoffigures
 
\end{document}